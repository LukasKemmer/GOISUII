\documentclass[ngerman, a4paper,12pt]{article}

%\usepackage[applemac]{inputenc} % Bei Benutzung von Apple-Betriebssystemen bitte durch ``\usepackage[applemac]{inputenc}'' ersetzen.
\usepackage[utf8]{inputenc}
\usepackage[T1]{fontenc}

\usepackage[ngerman]{babel} 
\usepackage{fixltx2e}
\usepackage{tabularx}
\usepackage{booktabs}
\usepackage{placeins}
\usepackage{eurosym}
\usepackage{amssymb,amsmath}

\usepackage{graphicx} 
\usepackage{color}
\definecolor{kit}{cmyk}{1,0,0.6,0}

\usepackage{hyperref}
\hypersetup{pdftoolbar=true,
            pdfmenubar=true,
            pdfpagemode=UseOutlines,
            bookmarksnumbered=true,
            linktocpage=true,
            colorlinks=false,
            %backref, % Entkommentieren, um zu sehen, ob alle Literaturstellen im Text zitiert werden.
            colorlinks=false
            }

\newtheorem{definition}{Definition}[section]
\newtheorem{satz}[definition]{Satz}
\newtheorem{lemma}[definition]{Lemma}
\newtheorem{korollar}[definition]{Korollar}
\newtheorem{proposition}[definition]{Proposition}
\newtheorem{bemerkung}[definition]{Bemerkung}
\newtheorem{beispiel}[definition]{Beispiel}
\newtheorem{problem}[definition]{Problem}
\newtheorem{Voraussetzung}[definition]{Voraussetzung}
\newtheorem{algorithmus}[definition]{Algorithmus}
\newtheorem{vermutung}[definition]{Vermutung}

\setlength{\parindent}{0pt}
\parskip1.5ex

\newcommand{\R}{\mathbb R} % Beispiel für die Definition eines eigenen Befehls

\begin{document}

\begin{flushleft}
\vspace*{-100pt}
\textbf{Institut f\"ur Operations Research \\
Prof. Dr. Oliver Stein \\}
Sommersemester 2018
\vspace*{15pt}
\end{flushleft}

\begin{flushright}
\vspace*{-80pt}
\includegraphics[scale=0.5]{kit_logo}
\vspace*{15pt}
\end{flushright}

\begin{center}
\textbf{Zweite Bonusübung zur Vorlesung \\
\emph{Globale Optimierung I}}        
\end{center}

\begin{table}[h]
	\centering
	\begin{tabularx}{\textwidth}{X X X X X}
		 & Vorname & Nachname & Matr.Nr. & Bachelor / Master \\
		\toprule
		1. Mitglied & Leon & Qadirie & 1720201 &  Master\\
		2. Mitglied & Lukas & Kemmer & 1725171 &  Master\\
		\bottomrule
	\end{tabularx}
\end{table}
\textbf{Aufgabe S2.1} \\

(a) Sei \par
(b) Sei \par
\textbf{Aufgabe S2.2} \\

(a) 
\par

(b) 
\par

(c) 
\par


(d) \par
\textbf{Aufgabe S2.3} \\

(a)
\par
(b) Sei 
\par
(c) 
\par
(d) 
\par
(e)
\par
\end{document}